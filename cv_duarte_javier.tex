%----------------------------------------------------------------------------------------
%	PACKAGES AND OTHER DOCUMENT CONFIGURATIONS
%----------------------------------------------------------------------------------------

\documentclass[11pt]{res}

%\usepackage{academicons}
\usepackage{palatino}%    Choose default roman font.  Others are times, pslatex, newcent, bookman, chancery
\usepackage{mathpazo}%  Match

\usepackage{helvet}%      Choose default sans serif

\usepackage{simplemargins}
\usepackage{slashed}
\usepackage[dvipsnames]{xcolor}
\usepackage{doi}


%\reversemarginpar % Move the margin to the left of the page
%\newcommand{\MarginText}[1]{\marginpar{\raggedleft\textsc{\small#1}}}
%% New command defining the margin text style
%\newcommand{\MarginText}[1]{\marginpar{\raggedleft\textbf{#1}}} % New command defining the margin text style
\newcommand{\MarginText}[1]{\section{#1}\vspace{10pt}}

%\usepackage[nochapters]{classicthesis} % Use the classicthesis style for the style of the document
%\usepackage[LabelsAligned]{currvita} % Use the currvita style for the layout of the document

%\renewcommand{\cvheadingfont}{\LARGE\color{Blue}} % Font color of your name at the top

\usepackage{hyperref} % Required for adding links	and customizing them
%\hypersetup{colorlinks, breaklinks, urlcolor=NavyBlue, linkcolor=NavyBlue, citecolor=Plum}
\hypersetup{colorlinks=false, breaklinks, pdfborder = {0 0 0}}
% Set link colors

\newcommand*\publistbasestyle{phys}
\usepackage[style=publist,
biblabel=brackets,
    %backend=biber,
 %   natbib,
  %  style=numeric,
   sorting=ddt,
   plauthorhandling=highlight,
   nameorder=given-family
]{biblatex}
 \plauthorname[Javier]{Duarte}
%\usepackage[%
%style=phys,%
%biblabel=brackets,%
%chaptertitle=false,%
%]
%{biblatex}

%\renewcommand*{\bibfont}{\small}
\DeclareSourcemap{
 \maps[datatype=bibtex,overwrite=true]{
  \map{
    \step[fieldsource=Collaboration, final=true]
    \step[fieldset=usera, origfieldval, final=true]
  }
 }
}

\renewbibmacro*{author}{%
  \iffieldundef{usera}{%
    \printnames{author}%
  }{%
    \printfield{usera} Collaboration% \printnames{author}%
  }%
}%

%%% --- The following two lines are what needs to be added --- %%%
\setcounter{biburllcpenalty}{7000}
\setcounter{biburlucpenalty}{8000}
% Name of your .bib file(s)
%\usepackage[utf8]{inputenc}
\addbibresource{bib_publications.bib}
\addbibresource{bib_refproceedings.bib}
\addbibresource{bib_proceedings.bib}
\addbibresource{bib_workinprogress.bib}
\addbibresource{bib_other.bib}

\newlength{\datebox}\settowidth{\datebox}{Spring 2011} % Set the width of the date box in each block

\newcommand{\NewEntry}[3]{\noindent\hangindent=2em\hangafter=0 \parbox{\datebox}{\small \textit{#1}}\hspace{1.5em} #2 #3 % Define a command for each new block - change spacing and font sizes here: #1 is the left margin, #2 is the italic date field and #3 is the position/employer/location field
\vspace{0.5em}} % Add some white space after each new entry

%\newcommand{\Description}[1]{\hangindent=2em\hangafter=0\noindent\raggedright\small{#1}\par\normalsize\vspace{.5em}} % Define a command for descriptions of each entry - change spacing and font sizes here

\newcommand{\Description[1]}{\noindent #1\vspace{-2pt}}

\usepackage[pazoGreek]{heppennames2}
\usepackage{ptdr-definitions}

\newcommand{\MR}{\ensuremath{M_\mathrm{R}}\xspace}
\newcommand{\MRz}{\ensuremath{M_\mathrm{R}^0}\xspace}
\newcommand{\Rtwo}{\ensuremath{\mathrm{R}^2}\xspace}
\newcommand{\Rtwoz}{\ensuremath{\mathrm{R}^2_0}\xspace}
%\usepackage{fullpage}


\newsectionwidth{10pt} % Stops section indenting

\setallmargins{1in}
%\setrightmargin{1in}

\begin{document}
%\thispagestyle{empty} % Stop the page count at the bottom of the first page

%----------------------------------------------------------------------------------------
%	NAME AND CONTACT INFORMATION SECTION
%----------------------------------------------------------------------------------------

\address{{\bf Javier M. Duarte} \\
Department of Physics, 0319\\
University of California San Diego\\
9500 Gilman Drive\\
La Jolla, CA 92093-0319}

\address{\\
office: Mayer Hall Addition 5513\\
email: \href{mailto:jduarte@ucsd.edu}{jduarte@ucsd.edu}\\
web: \href{https://jduarte.physics.ucsd.edu/}{https://jduarte.physics.ucsd.edu}\\
phone: (858) 246-4980\\
%Publons: \href{https://publons.com/researcher/AAA-5414-2020/}{\aiPublons~AAA-5414-2020}\\
%ORCID: \href{https://orcid.org/0000-0002-5076-7096}{\aiOrcid~0000-0002-5076-7096}
}

\newcommand{\DOI}[1]{\href{https://doi.org/#1}{doi:#1}}
\begin{resume}

%----------------------------------------------------------------------------------------
%	EDUCATION
%----------------------------------------------------------------------------------------

\MarginText{Education}

\textbf{California Institute of Technology} (\textit{2011--2016})\\
M.S., Physics, 2015\\
Ph.D., Physics, 2016\\
Dissertation:
    \href{https://doi.org/10.7907/Z9GX48JV}{Naturalness
        confronts nature: searches for supersymmetry with the CMS
        detector in $\Pp\Pp$ collisions at $\sqrt{s} = 8$ and $13\TeV$}\\
Advisor: Professor Maria Spiropulu

\textbf{Massachusetts Institute of Technology} (\textit{2006--2010})\\
S.B., Mathematics, 2010\\
S.B., Physics, 2010\\
Senior thesis: \href{http://hdl.handle.net/1721.1/61255}{Exotic antineutrino oscillations
  ($\bar\nu_\Pe\to\bar\nu_{\slashed{\Pe}}$) in Double Chooz}\\
Advisor: Professor Janet Conrad


%----------------------------------------------------------------------------------------
%	PROFESSIONAL EXPERIENCE
%----------------------------------------------------------------------------------------

\MarginText{Professional Experience}
Assistant Professor of Physics at UC San Diego, La Jolla, CA (\textit{2019--Present})\\
\href{http://www.fnal.gov/pub/forphysicists/fellowships/leon_lederman/index.html}{Lederman
  Fellow} at Fermilab, Batavia, IL (\textit{2016--2019})\\
Technical Instructor in
\href{http://web.mit.edu/8.13/www/index.shtml}{Junior Lab} at MIT, Cambridge, MA (\textit{2010--2011})


\MarginText{Grants and External Funding}
PI of \href{https://science.osti.gov/-/media/early-career/pdf/FY20_DOE_SC_Early_Career_Research_Program_Abstracts.pdf}{DOE Early Career Award for ``Real-Time Artificial Intelligence for Particle Reconstruction and Higgs Physics''} (\textit{2020--2025})\\
Co-PI of \href{https://nsf.gov/awardsearch/showAward?AWD_ID=2005369}{NSF Award for ``Category II: Exploring Neural Network Processors for AI in Science and Engineering''} (\textit{2020--2021})\\
Key Personnel for ``Investigating Heterogeneous Computing at the Large Hadron Collider'' Phase-II sub-award of \href{https://www.nsf.gov/awardsearch/showAward?AWD_ID=1904444}{Internet2 NSF Grant ``Exploring Clouds for Acceleration of Science (E-CAS)''} (\textit{2020-2021})\\
\href{https://pamspublic.science.energy.gov/WebPAMSExternal/Interface/Common/ViewPublicAbstract.aspx?rv=1f7d4729-6f93-40bd-a55f-c108545b1ea9&rtc=24&PRoleId=10}{DOE QuantISED Award for ``Quantum Machine Learning and Quantum Computation Frameworks for HEP (QMLQCF)''} (\textit{2018--2020})\\
\href{https://ldrd.fnal.gov/subdir/FNAL-LDRD-2019-017-D1.pdf}{Fermilab LDRD Award for ``Graph Neural Networks for Accelerating Calorimetry and Event Reconstruction''} (\textit{2019--2021})\\
\href{https://ldrd.fnal.gov/subdir/FNAL-LDRD-2019-027-D1.pdf}{Fermilab LDRD Award for ``Accelerator Control with Artificial Intelligence''} (\textit{2019--2021})\\

%----------------------------------------------------------------------------------------
%	FELLOWSHIPS AND AWARDS
%----------------------------------------------------------------------------------------

\MarginText{Fellowships and Awards} %, and Grants}
\href{https://science.osti.gov/-/media/early-career/pdf/FY20_DOE_SC_Early_Career_Research_Program_Abstracts.pdf}{DOE Early Career Award} (\textit{2020})\\
William A. Lee Chancellor's Endowed Junior Faculty Fellow II (\textit{2019--Present})\\
\href{http://lpc.fnal.gov/fellows/2019/Javier_Duarte.shtml}{LHC Physics Center Distinguished Researcher} (\textit{2019})\\
\href{http://www.fnal.gov/pub/forphysicists/fellowships/leon_lederman/index.html}{Fermilab Lederman Fellowship} (\textit{2016--2019})\\
\href{http://pma.caltech.edu/research-and-academics/physics/physics-prizes-awards/france-a-cordova-graduate-student-fund-recipients}{Tombrello Scholar} sponsored by the France A. C\'{o}rdova Graduate Student Fund (\textit{2015})\\
\href{https://www.nsfgrfp.org/}{NSF Graduate Research Fellowship} (\textit{2011--2014})\\
%\Description{\textit{2012} \hspace{1.5em} Isidor I. Rabi Award at the \href{http://www.emfcsc.infn.it/issp2012/index.html}{International School of Subnuclear Physics}}\\
\href{http://www.gmsp.org/}{Gates Millenium Scholar} sponsored by Hispanic Scholarship Fund (\textit{2006--2014})

%----------------------------------------------------------------------------------------
%	Research
%----------------------------------------------------------------------------------------

\MarginText{Selected Research Experience}

\textbf{Quantum Computing for HEP.} Studies of quantum machine learning and
quantum computation frameworks for high energy physics~\cite{Zlokapa:2019tkn}.
(\textit{2018--Present}).

\textbf{Physics Management.} Co-convener of CMS Exotica Jets+X subgroup
(\textit{2018--Present}).
%\begin{itemize}
%\item Coordinate, review, and define LHC Run 2 ``legacy''
%  publication strategy for over 20 different physics analysis efforts.
%\end{itemize}

\textbf{Boosted Higgs.} Lead author of search for a highly Lorentz boosted Higgs boson
  decaying to a bottom quark-antiquark pair using full Run 2 data~\cite{Sirunyan:2020hwz} (\textit{2017--Present}).

\textbf{Machine Learning / Coprocessors.} Development of \href{https://github.com/hls-fpga-machine-learning/SonicCMS}{Services for Optimal Network Inference on Coprocessors (SONIC)}~\cite{Krupa:2020bwg,neurips2019_sonic,Duarte:2019fta} (\textit{2018--Present}).
%\begin{itemize}
%\item Prepared, trained, validated, and tested the ResNet-50 convolutional neural network for top quark
%  jet identification at the LHC on different CPU, GPU, and FPGA platforms, including Microsoft Project Brainwave.
%\end{itemize}

\textbf{Machine Learning / Trigger.} Development of
\href{https://fastmachinelearning.org/hls4ml/}{\texttt{hls4ml}}
tool for creating low-latency FPGA-based firmware
  implementations of machine learning algorithms~\cite{DiGuglielmo:2020eqx,Summers:2020xiy,neurips2019_hls4ml,Duarte:2018ite} (\textit{2017--Present}).
%\begin{itemize}
%\item Prepared benchmark machine learning algorithms, including
%  fully connected, convolutional, recurrent, and graph neural networks.
%\item Designed and implemented firmware using High-Level Synthesis for multiple
%  machine learning algorithms, including convolutional and graph neural networks.
%\end{itemize}

\textbf{Trigger.} R\&D, including firmware development and hardware demonstration, for
the CMS Global Correlator Trigger for the Phase-2 upgrade of the
Level-1 trigger~\cite{collaboration:2714892} (\emph{2017--Present}).
%\begin{itemize}
%\item Firmware and feasibility studies of different ``regionizing'' and
%  ``time-multiplexing'' scenarios for optimal interface between upstream
%  detectors and Global Correlator Trigger.
%\end{itemize}

\textbf{Machine Learning / Higgs Tagging.} Development of a mass-decorrelated deep neural network tagger
  for identifiying boosted Higgs bosons decaying to $\bbbar$ and $\ccbar$ inside and outside of  CMS~\cite{Moreno:2019neq,Moreno:2019bmu,neurips2019_hbb,CMS-DP-2018-046}
  (\textit{2017--Present}).
%\begin{itemize}
%\item Developed, trained using simulated data, and
%  deployed a deep neural network for identifiying boosted Higgs
%  bosons decaying to $\bbbar$ and $\ccbar$ over a wide range of
%  masses.
%\item Led a team performing studies in data to commission the neural network.
%\item Developed new interaction network~\cite{Moreno:2019bmu,Moreno:2019neq} methods to further improve background
%  rejection as well as coordinated the release of the simulated data through the CERN Open Data Portal~\cite{opendata}.
%\end{itemize}


\textbf{Boosted Higgs.} Co-author of first search for a highly Lorentz boosted Higgs boson
  decaying to a bottom quark-antiquark pair using 2016 data, published in
  PRL~\cite{Sirunyan:2017dgc}; Adapted analysis for interpretation for differential gluon fusion Higgs boson $\pt$ measurement~\cite{Sirunyan:2018sgc}; Combination of this result with other channels led to an observation of $\PH(\bbbar)$ decay~\cite{Sirunyan:2018kst} and other measurements~\cite{Sirunyan:2018koj} (\textit{2016--2018}).
%\begin{itemize}
%\item Performed the first sensitivity studies, processed the data, defined and validated
%  the background estimation method, and extracted the Higgs signal.
%\item Delivered pre-approval and approval presentations within
%  the CMS collaboration and co-authored the paper communicating our
%  results, published in PRL.
%\item Adapted analysis and interpretation for differential gluon
%  fusion Higgs boson $\pt$ measurement~\cite{Sirunyan:2018sgc}.
%\item Combination of this result with other channels led
%  to an observation of $\PH(\bbbar)$ decay~\cite{Sirunyan:2018kst} and other measurements~\cite{Sirunyan:2018koj}.
%\end{itemize}

\textbf{Boosted Dijet Resonances.} Co-leader of analysis group searching for exotic, light spin-$1$ and
  spin-$0$ particles decaying to
  quarks~\cite{Sirunyan:2019vxa,Sirunyan:2019sgo,Sirunyan:2018ikr,Sirunyan:2017nvi} (\textit{2017--2019}).
%\begin{itemize}
%\item I lead a group of professors, postdocs, and graduate students
%  from Fermilab, MIT, Rutgers, Johns Hopkins, and Brown, which use shared techniques like jet
%  substructure and machine learning algorithms, to perform searches for new, light, boosted particles.
%\item Within this group, I am the lead author of a search, which
%  generalizes and extends the boosted Higgs analysis, for exotic, light
%  spin-$0$ particles decaying to bottom quark-antiquark pairs in the
%  mass range between $50$ and $350\GeV$~\cite{Sirunyan:2018ikr}.
%\end{itemize}

\textbf{Dijet Resonances.} Co-leader of dijet resonance search group, including data scouting, wide resonance, and $\PQb$-tagged resonance searches~\cite{Sirunyan:2019pnb,Sirunyan:2019vgj,CMS-PAS-EXO-17-026,Duarte:2018bsd,Sirunyan:2018xlo,Sirunyan:2016iap} (\textit{2016--2018}).
%\begin{itemize}
%\item Developer of the 2017 CMS data scouting triggers, streams, and data sets.
%\item Developer of statistical analysis framework for dijet resonance searches.
%\end{itemize}

\textbf{Trigger.} Level-1 and high-level (software) trigger development for Higgs decaying to
bottom quark-antiquark pairs produced in association with a $\PZ$ boson
decaying to neutrinos~\cite{Sirunyan:2018kst} (\textit{2016--2017}).

\textbf{Trigger.} Tested 3D Vertically-Integrated Pattern Recognition
Associate Memory (VIPRAM) chips for the Phase-2 upgrade of the CMS Level-1
track trigger (\textit{2016--2017}).

%Lead analyzer for dijet searches at 13\TeV using data
 % scouting (\textit{2016--2017}).
%\begin{itemize}
%\item I lead the dijet resonance search with $12.9\fbinv$ CMS data set at
 % $13\TeV$ including the first dark matter
 % mediator interpretation from CMS~\cite{Sirunyan:2016iap}.
%\end{itemize}

\textbf{SUSY Analysis.} Lead analyzer for supersymmetry (SUSY) searches using
  razor variables at 8 and 13\TeV~\cite{Khachatryan:2016epu,Duarte:2016wnw,Khachatryan:2015pwa,Duarte:2014soa} (\textit{2012--2016}).
%\begin{itemize}
%\item Led the inclusive searches for supersymmetry in events at $8$ and $13\TeV$ using razor variables and
%  a fully data-driven background estimation method resulting in the first
%  branching-ratio-independent bounds on supersymmetric particles~\cite{Khachatryan:2015pwa,Khachatryan:2016epu}.
%\item I designed and validated the procedure of using a
%  three-dimensional maximum likelihood fit (with observables $\MR$, $\Rtwo$,
 % and the number of b-tagged jets) in a
 % background-enriched sideband to estimate the standard model
 % backgrounds in a signal-sensitive region.
%\item Produced first set of Monte Carlo samples using \MADGRAPH
% and \PYTHIA for varying SUSY branching fraction
% choices in order to set a more universal bound on the SUSY particle
% masses.
%\item The resulting constraint on SUSY particle masses was the first branching-ratio-independent limit
 % produced by CMS.
%\end{itemize}

\textbf{SUSY Phenomenology.} Development of a simplified model~\cite{Duarte:2017bbq}
  for bottom-squark-mediated Higgs+jet production used to interpret a
  CMS search for supersymmetry with $\PH\to\gamma\gamma$
  decays~\cite{Sirunyan:2017eie} (\textit{2015--2016}).

\textbf{Timing Detector.} Measurements of the timing properties of
  LYSO-based calorimeters at Fermilab and CERN test beam facilities, demonstrating that a timing
  resolution of approximately 30\unit{ps} is achievable~\cite{Bornheim:2017gql,testbeam,Anderson:2016tiu,Anderson:2016ygg,NIMA,Bornheim_2015,Anderson:2015tia}
  (\textit{2014--2016}).

%\item Developed software in \textsc{C++} and python to perform the readout and timing analysis of
 % the pulses from specialized PSI DRS4 chips and other oscilloscopes.
%\item Together with collaborators, demonstrated that a timing
%  resolution of approximately 30~\unit{ps} is achievable~\cite{NIMA}.

\textbf{SUSY Triggers.} Development and implementation of high-level
triggers for selecting SUSY and dark matter candidate events using
razor variables for Run 2 of the LHC at 13 TeV  (\textit{2014--2016}).

%\item I developed a novel trigger to select SUSY candidate events
%  containing a Higgs boson decaying to a bottom quark-antiquark pair,
 % inspired by an excess in an 8 TeV SUSY search in the diphoton final state~\hyperlink{ref:PAS14007}{[3]}
%\item Developed software in \textsc{C++} and python for validating the alignment and
 % calibration conditions of the detector during data-taking and coordinated
  %the trigger development and deployment for all alignment and calibration activities

%\Description{Generation of Monte Carlo Events for the Search for SUSY
 % at $\sqrt{s}=7$ TeV (2011--2012)
%\begin{itemize}
%\item I generated Monte Carlo events and ran the CMS Fast Simulation
  %to produce a large sample of simulated SUSY events ($\sim10$ million) in the
 % context of the Constrained Minimal Supersymmetric Standard
 % Model (CMSSM).
%\item These samples were used in the interpretation of the
 % search for SUSY using razor variables at $\sqrt{s}=7$ TeV to place
 % constraints on the $m_0$ and $m_{1/2}$ parameters of the CMSSM~\hyperlink{ref:7TeVPRL}{[8]}~\hyperlink{ref:7TeVPRD}{[9]}~\hyperlink{ref:cmssm}{[10]}.
%\end{itemize}}

% \Description{CMS Open Data (2015--Present)
% \begin{itemize}
% \item I created a \href{https://github.com/jmduarte/SUSYBSMAnalysis-RazorFilter}{public
%       package on GitHub} in \textsc{C++} and python to read the CMS open data and produce a
%   simple output \textsc{ROOT} TTree or CSV file with jet four-vectors and kinematic variables, such
%   as $M_R$, $R^2$, $\HT$, and $\MET$, useful for SUSY
%   searches.
% \item The goal is to make the CMS data more accessible to the
%   greater physics community.
% \end{itemize}}


%\Description{CMS Data Analysis School (2015)
%\begin{itemize}
%\item I co-led one of the long exercises at the \href{https://indico.cern.ch/event/346968/}{CMS Data Analysis
 % School} (January 12-16, 2015) for five students.
%\item The exercise consisted of developing a
 % search for supersymmetry using razor variables and jet substructure
 % in final states with a boosted W boson and computing the sensitivity
 % to top squark production in CMS for the 13 TeV Run.
%\end{itemize}}

%\NewEntry{2009--2010}{Undergraduate Researcher in the MIT Double Chooz group}%

%\Description{Double Chooz Sensitivity to Sterile Neutrino Oscillations (2009--2010)
%\begin{itemize}
%\item I analyzed simulated reactor neutrino oscillation data
 % to determine the sensitivity of the Double Chooz experiment to
 % observe neutrino oscillations in the context of two additional
 % sterile neutrino families.
%\end{itemize}}

%\NewEntry{2009}{Undergraduate Researcher in the University of
 % Michigan ATLAS group}

%\Description{GMSB SUSY Discovery Potential (2009)
%\begin{itemize}
%\item I performed an analysis to determine the discovery potential of
 % the ATLAS detector for Gauge-Mediated SUSY Breaking (GMSB), targeting neutralino production
 % with a $Z\to\ell\ell$ in the final state.
%\end{itemize}}


%\NewEntry{2007}{Undergraduate Researcher in the Columbia
%  MiniBooNE group}
%
%\Description{Off-axis Neutrino Beam Simulations (2007)
%\begin{itemize}
%\item I investigated the possibility (through a variety of simulations) of
 % observing the off-axis NuMI beam in the
 % MiniBooNE detector to make the an off-axis measurement of
 % NuMI beam neutrino flux.
%\end{itemize}}

%----------------------------------------------------------------------------------------
%	PUBLICATIONS
%----------------------------------------------------------------------------------------
%\clearpage
\MarginText{Selected Publications, Reports, and Conference Proceedings}

Selected publications, reports, and proceedings in which I made a leading or significant contribution are listed here.\nocite{Albertsson:2018maf}

\vspace{10pt}
\printbibliography[heading=none,sorting=ynt,keyword={career}]

%----------------------------------------------------------------------------------------
%	TEACHING
%----------------------------------------------------------------------------------------

\MarginText{Teaching}

\textbf{Introductory Physics.} Lead instructor for
\href{https://jduarte.physics.ucsd.edu/phys2c/index.html}{Physics 2C}:
Fluids, Waves, Thermodynamics, and Optics for 350 undergraduate students
(Winter \textit{2020})

\textbf{Graduate Seminar.} Guest speaker for Physics 261:
Seminar on Physics Research at UC San Diego (Winter \textit{2020})

\textbf{Undergraduate Seminar.} Guest speaker for Physics 191:
Undergraduate Seminar on Physics (Fall \textit{2019})

\textbf{Physics Tutorial.} Lead facilitator of the
\href{https://indico.cern.ch/event/628146/}{first} and second annual \href{https://indico.cern.ch/event/726984/}{CMS Machine Learning
    Tutorial} attended by over 60 participants (\textit{2017--Present})

\textbf{Teacher Training.} Facilitator for
  \href{http://eddata.fnal.gov/lasso/program_search/show_workshopID_new.lasso?event_id=435}{Taking  Research into Your Classroom} Workshop at Waubonsee Comunity College (\textit{2017})

\textbf{Teaching Assisting.} Teaching Assistant in statistical and quantum mechanics at Caltech (\textit{2011--2012})

\textbf{Physics Teaching.} Technical Instructor in \href{http://web.mit.edu/8.13/www/index.shtml}{MIT Junior Lab}, teaching third-year
  undergraduate physics students and maintaining the experiments (\textit{2010--2011})


%----------------------------------------------------------------------------------------
%	OUTREACH
%----------------------------------------------------------------------------------------

  \MarginText{Outreach}

\textbf{Public Lecture.} Invited faculty speaker for \href{http://ypp.ucsd.edu/}{Young Physicists Program} at UC
San Diego (\textit{2020}).

\textbf{Outreach Management.} Co-director of \href{http://saturdaymorningphysics.fnal.gov/}{Saturday
  Morning Physics} and lectuer on \href{http://saturdaymorningphysics.fnal.gov/fall-session-2018/}{Symmetry, Antimatter, and Supersymmetry} at Fermilab (\textit{2018--2019})

\textbf{Outreach Coordination.} On-site coordinator for \href{http://saturdaymorningphysics.fnal.gov/}{Saturday Morning Physics} at Fermilab (\textit{2016--2018})

\textbf{Diversity Programs.} Mentor in the \href{http://diversity.fnal.gov/sist/}{SIST} internship program at Fermilab (\textit{2018})

\textbf{Diversity Programs.} Graduate student ambassador for the \href{http://diversity.fnal.gov/fshpe/}{Fermilab SHPE chapter} (\textit{2018})

\textbf{Diversity Programs.} Member of the \href{http://diversity.fnal.gov/target/}{TARGET} program committee at Fermilab (\textit{2017--Present})

\textbf{Physics Teaching / Diversity Programs.} Residential Facilitator for \href{http://ome.mit.edu/programs-services/program-overview}{MIT
  Interphase EDGE program} (\textit{2010})

%----------------------------------------------------------------------------------------
%	PRESENTATIONS
%----------------------------------------------------------------------------------------

\MarginText{Selected Conference, Workshop, and Seminar Presentations}

\href{https://indico.fnal.gov/event/22961/}{High Energy Physics Division Seminar [Remote]}. April 8, 2020. Argonne National Laboratory, Lemont, IL, USA.

\href{https://ml4physicalsciences.github.io/files/NeurIPS_ML4PS_2019_64.pdf}{``Accelerated Machine Learning as a Service for
Particle Physics Computing.''} Machine Learning and the Physical Sciences Workshop, NeurIPS 2019. December 14, 2019. Vancouver, Canada.

\href{https://ml4physicalsciences.github.io/files/NeurIPS_ML4PS_2019_71.pdf}{``Interaction networks for the identification of Higgs boson decays to bottom quark-antiquark pairs.''} Machine Learning and the Physical Sciences Workshop, NeurIPS 2019.  December 14, 2019. Vancouver, Canada.

\href{https://ml4physicalsciences.github.io/files/NeurIPS_ML4PS_2019_74.pdf}{``Low-latency machine learning inference on FPGAs.''} Machine Learning and the Physical Sciences Workshop, NeurIPS 2019.  December 14, 2019. Vancouver, Canada.

\href{https://physics.drupal.ku.edu/calendar/colloquia#/?i=2}{Department of Physics and Astronomy Colloquium}. November 11, 2019. University of Kansas, Lawrence, KS, USA.

\href{http://physics.bu.edu/events/show/2204}{High Energy Experiment Seminar}. October 10, 2019. Boston University, Boston, MA, USA.

\href{https://escience2019.sdsc.edu/program}{``Machine
  learning on FPGAs for low-latency and high-throughput inference.''}
eScience 2019. September 24–27, 2019. San Diego, CA, USA.

\href{https://indico.cern.ch/event/793125/contributions/3495251/}{``Deep learning on FPGAs tutorial.''} 1st Real Time Analysis Workshop. July 15-26, 2019. Institute Pascal, Universit\'{e} Paris-Saclay, Saint Aubin, France.

\href{https://indico.cern.ch/event/687651/contributions/3428206/}{``Machine
  learning using CERN Open Data.''} LHCP 2019. May
  20-25, 2019. Benem\'{e}rita Universidad Aut\'{o}noma de Puebla, Puebla, Mexico.

\href{https://indico.cern.ch/event/687651/contributions/3426898/}{``Dark
  sector searches in CMS.''} LHCP 2019. May
  20-25, 2019. Benem\'{e}rita Universidad Aut\'{o}noma de Puebla, Puebla, Mexico.

\href{https://indico.cern.ch/event/742793/contributions/3274392/}{``FPGA-accelerated
  machine learning inference for particle physics.''} Connecting the
Dots 2019. April 2, 2019. Valencia, Spain.

\href{https://universityofchicago.hosted.panopto.com/Panopto/Pages/Viewer.aspx?id=66ca09d7-74c1-4b12-bb57-a9fa01046cdf}{``Unlocking
  the potential of LHC data: boosted Higgs and deep learning.''}
Particle Physics Special Seminar. February 20, 2019. University of Chicago, Chicago, IL, USA.

\href{http://theory.fnal.gov/events/event/results-from-cms-18/}{``Boosted
  Higgs couplings and dark mediators with deep learning in CMS.''}
Joint Experimental-Theoretical Physics Seminar (Wine \& Cheese). December 14, 2018. Fermilab, Batavia, IL, USA.

\href{https://indico.cern.ch/event/745718/contributions/3211982/}{``Heavy
  flavour identification for boosted resonances and large cone jets in
  CMS.''} Machine Learning for Jet Physics (ML4Jets) 2018. November
14-16, 2018. Fermilab, Batavia, IL, USA.

\href{https://www.physicsandastronomy.pitt.edu/events/hep-seminar-javier-duarte-fermilab}{``Boosted
    Higgs, dark matter, and deep learning.''} High Energy Physics
  Seminar. October 3, 2018. University of Pittsburgh, Pittsburgh, PA, USA.

\href{https://indico.cern.ch/event/697988/contributions/3055990/}{``\texttt{hls4ml}:
    Deploying Deep Learning on FPGAs for L1 trigger and Data
    Acquisition.''} Topical Workshop on Electronics for
  Particle Physics (TWEPP) 2018. September 17-21, 2018. KU Leuven Campus Carolus, Antwerp, Belgium.

\href{https://conferences.lbl.gov/event/137/session/27/contribution/354}{``Searches
    for Dark Matter Mediators with the CMS Detector.''} Conference on the Intersections of Particle and Nuclear
  Physics (CIPANP) 2018. May 29 - June 3, 2018. Hyatt Regency Indian Wells
  Conference Center, Indian Wells, CA, USA.

\href{https://indico.fnal.gov/event/16908/}{``Fast inference of deep neural networks in FPGAs for particle physics.''} Research Techniques Seminar. April 24, 2018. Fermilab, Batavia, IL, USA.

\href{https://indico.cern.ch/event/658267/contributions/2881127/}{``Fast Reconstruction and Data Scouting.''} Connecting the Dots 2018. March 20-22,
  2018. University of Washington, Seattle, WA, USA.

\href{http://www.thphys.uni-heidelberg.de/~higgs/talks/duarte.pdf}{``Boosted Higgs in CMS.''} Higgs Couplings 2017. November 6-10, 2017. Heidelberg
  University, Heidelberg, Germany.

\href{https://indico.hep.caltech.edu/indico/conferenceDisplay.py?confId=149}{``Unlocking the potential of CMS data: boosted Higgs, low-mass dijet resonances, and data scouting.''} High Energy Physics Seminar. October 30, 2017. Caltech, Pasadena, CA, USA.

\href{https://indico.cern.ch/event/615891/contributions/2666361/}{``Search
    for low-mass dijet resonances.''} TeVPA 2017. August 7-11,
  2017. Columbus, OH, USA.

\href{https://indico.fnal.gov/contributionDisplay.py?sessionId=14&contribId=38&confId=11999}{``Inclusive
  search for boosted SM Higgs bosons using H to bb decays with the CMS
  detector at 13 TeV.''} APS DPF 2017.  July
  31 - August 4, 2017. Fermilab, Batavia, IL, USA.

\href{https://indico.cern.ch/event/649575/}{``Inclusive
    Higgs boson search using $\PH\to\bbbar$ decays''} Collider Cross
  Talk. July 20, 2017. CERN, Geneva, Switzerland.

\href{https://indico.fnal.gov/contributionDisplay.py?sessionId=6&contribId=54&confId=13497}{``Introduction
    to CMS open data for boosted object tagging with machine learning
    applications.''} Data Science at High Energy Physics 2017. May
  8-12, 2017. Fermilab, Batavia, IL, USA.

\href{https://indico.cern.ch/event/540843/contributions/2464658/}{``Dijet
    and boosted dijet searches for low-mass resonances.''} New Physics Interpretations at the LHC 2. April
  5-7, 2017. Argonne National Laboratory, Lemont, IL, USA.

\href{https://indico.cern.ch/event/592671/contributions/2401900/}{``Data
    scouting and boosted dijets: searches for low-mass dijet
    resonances.''} Digging Deeper at LHC Run II. February
  23-25, 2017. University of Pittsburgh, Pittsburgh, PA, USA.

\href{https://indico.cern.ch/event/443176/contributions/2148316/}{``Inclusive
    searches for SUSY using the razor variables in CMS.''} SUSY 2016. July 3-8, 2016. The University of Melbourne,
  Melbourne, Australia.

\href{https://indico.cern.ch/event/443176/contributions/2154549/}{``Searches
    for BSM physics in dijet and multijet final states at CMS.''} SUSY 2016. July 3-8, 2016. The University of Melbourne,
  Melbourne, Australia.

\href{https://indico.cern.ch/event/331032/contributions/1720249/}{``Third
    generation SUSY searches with the CMS detector.''} SUSY 2015. August 23-29, 2015. Lake Tahoe,
  CA, USA.

\href{https://indico.ific.uv.es/indico/contributionDisplay.py?sessionId=24&contribId=289&confId=2025}{``Inclusive
    SUSY searches at CMS.''} ICHEP 2014. July 2-9, 2014. Valencia
  Conference Centre, Valencia, Spain.

\href{https://indico.cern.ch/event/279518/contributions/634785/}{``Search
    for SUSY with the razor variables at CMS.''} LHCP 2014. June
  2-7, 2014. Columbia University, New York, NY, USA.

\href{https://indico.cern.ch/event/252857/contributions/1579321/}{``Searches
    for Supersymmetry with the CMS Detector.''} 2nd
  High Energy Physics in the LHC Era. December 16-20,
  2013. Universidad T\'{e}cnica Federico Santa Mar\'{i}a,
  Valpara\'{i}so, Chile.

\href{https://indico.cern.ch/event/261650/contributions/586374/}{``Razor
    Search for SUSY in CMS at $\sqrt{s}=8$ TeV.''} Kinematic Variables for
  New Physics at the LHC. October 4-6, 2013. Caltech, Pasadena, CA, USA.

\href{https://indico.cern.ch/event/218693/contributions/1520333/}{``BSM Likelihoods in CMS.''} Likelihoods for
    the LHC Searches Workshop. January 21-23, 2013. CERN, Geneva,
    Switzerland.


%----------------------------------------------------------------------------------------
%	COMMITTEES
%----------------------------------------------------------------------------------------

\MarginText{Service and Committee Work}

French National Research Agency (ANR) external reviewer (\textit{2020})

Scientific organizing committee for \href{http://www.ncsa.illinois.edu/Conferences/AcceleratedAINCSA/}{Accelerated Artificial Intelligence for Big-Data Experiments Conference} (2020) at NCSA

Nuclear Instruments and Methods in Physics Research Section A (NIMA): Accelerators, Spectrometers, Detectors and Associated Equipment peer reviewer (\textit{2020})

Graduate admissions committee for UC San Diego Physics Department (\textit{2019--2020})

European Physics Journal C (Eur. Phys. J. C) peer reviewer (\textit{2019})

European Science Foundation (ESF) external reviewer (\textit{2018--Present})

Local organizing committee for \href{https://indico.cern.ch/e/fml}{Fast ML} workshop at Fermilab LPC (\textit{2019})

Local organizing committee for \href{https://indico.cern.ch/e/ml4jets2018}{ML4Jets 2018} workshop at Fermilab LPC (\textit{2018})

\href{http://diversity.fnal.gov/target/}{TARGET} program committee at Fermilab (\textit{2017--2019})

\href{http://lpc.fnal.gov/programs/topic/}{LPC Topic of the Week} committee chair at Fermilab (\textit{2017--2018})

Analysis review committees for several CMS analyses: a search for new physics with dijet angular distributions~\cite{Sirunyan:2018wcm}, a search for fully hadronic $\ttbar\PH(\bbbar)$~\cite{Sirunyan:2018ygk}, a search for strongly interacting massive particles, and two searches for electroweak production of a vector-like top quark partner in the fully-hadronic final state~\cite{Sirunyan:2019xeh}.

Local organizing committee for \href{http://dshep.fnal.gov}{DS@HEP 2017} workshop at Fermilab and development of public data set based on CMS open data for boosted object tagging with machine learning applications

Local organizing committee for \href{https://indico.cern.ch/event/639314/}{LHC Reinterpretation 2017} workshop at Fermilab LPC


\MarginText{Professional Memberships}

American Physical Society member (\textit{2020})

Society of Hispanic Professional Engineers (\textit{2018--2019})


% %----------------------------------------------------------------------------------------
% % CMS INTERNAL WORK
% %----------------------------------------------------------------------------------------

% \vspace{1em}
% \MarginText{CMS Internal Work and Leadership}
% \vspace{-1em}


% \Description{
% \begin{itemize}
% \item CMS Level-3 Exotica Jets+X subgroup co-convener (\textit{2018--Present}).
% \item Contact person for search for boosted exotic spin-$0$ particles
%   decaying to bottom quark-antiquark pairs based on 2016 CMS data~\cite{Sirunyan:2018ikr}.
% \item Pre-approval and approval presentations for search for boosted Higgs
%   decaying to bottom quark-antiquark pairs based on 2016 CMS data~\cite{Sirunyan:2017dgc}.
% \item Pre-approval presentation for 13 \TeV razor SUSY
%   search based on 2015 CMS data~\cite{Khachatryan:2016epu}.
% \item Pre-approval and approval presentations for 8 \TeV razor SUSY
%   search based on 2012 CMS data~\cite{Khachatryan:2016epu}.
% \item CMS Level-3 leadership position of the Alignment
%   and Calibration (AlCa) High Level Trigger (HLT) contact (\textit{2013--2015}).
% \item Developer of the CMS data scouting triggers, streams,
%   and data sets (\textit{2017--Present}).
% \item Development of a mass-decorrelated deep neural network tagger
%   for identifying boosted Higgs bosons decaying to $\bbbar$ or
%   $\ccbar$ in CMS~\cite{CMS-DP-2018-046} (\textit{2017--Present}).
% \item Generation of Monte Carlo samples using \MADGRAPH and \PYTHIA
%   for varying SUSY branching fraction choices in order to set a more
%   universal bound on the SUSY particle masses (\textit{2012--2013}).
% \end{itemize}}

%----------------------------------------------------------------------------------------
% STUDENTS
%----------------------------------------------------------------------------------------

\MarginText{Student Supervision}

Supervision of Graduate Students (\textit{2016--Present}).

\begin{itemize}
\item Raghav Kansal (UC San Diego). Graph-based Generative Adversarial
  Networks for HGCal simulation (\textit{2019--Present}).
\item Martin Kwok (Brown). Boosted Higgs boson search (\textit{2018--Present}).
\item Michael Krohn (CU Boulder). Boosted Higgs boson search, coupling
  measurement, and trigger development (\textit{2017--2018}).
\item Sean-Jiun Wang (University of Florida). Development and
  monitoring of triggers for the Higgs boson produced in association with a
  $\PZ$ boson decaying to neutrinos (\textit{2017--2018}).
\item Andrzej Novak (RWTH Aachen University). Development of deep neural
  networks for boosted Higgs identification in CMS (\textit{2017--Present}).
\item Jiajing Mao (Caltech). Data scouting trigger stream development (\textit{2016--Present}).
\item Giulia D'Imperio and Federico Preiato (Sapienza University of Rome). Dijet searches (\textit{2016}).
\end{itemize}

Supervision of Undergraduate Students (\textit{2012--Present}).

\begin{itemize}
\item Steven Tsan (UC San Diego). Unsupervised or semi-supervised anomaly detection algorithms for high energy physics. (\textit{2020--Present}).
\item Vesal Razavimaleki (UC San Diego). Implementation of graph neural networks on FPGAs. (\textit{2019--Present}).
\item Eric Moreno (Caltech). Development of interaction and graph neural networks for boosted jet tagging with CMS open data. (\textit{2018--Present}).
\item Sydney Jenknins (University of Chicago). Compression and firmware implementation of interaction and graph neural networks for charged particle tracking at the LHC (\textit{2018}).
\item Eric Scotti (Brown University). Development of deep neural networks for boosted Higgs identification in CMS (\textit{2017--2018}).
\item Irene Li (University of Cambridge). Supervised and unsupervised machine learning methods for SUSY searches at the LHC (\textit{2015}).
\item Jared Filseth (Caltech). Measurement of single-lepton and razor trigger efficiencies for LHC Run 2 (\textit{2015}).
\item Ann Wang (Caltech). Trigger and search strategy for new Higgs-aware SUSY models at the LHC (\textit{2013--2015}).
\item Edward Garza (Caltech). Optimization of megajet clustering for razor SUSY searches at the LHC (\textit{2014}).
\item Gautam Uphadya (Caltech). New kinematic variables for gluino-mediated SUSY production (\textit{2013--2014}).
\item Natalie Harrison (University of Chicago). EFT Interpretation of the razor search for DM production at the LHC (\textit{2013}).
\item Anthony Lutz (Caltech). Generator-level validation of Monte Carlo samples for simplified SUSY models with mixed final states (\textit{2013}).
\item Nikita Sirohi (Caltech). Monte Carlo generation of simplified models for DM searches at LHC Run 2 (\textit{2013}).
\item Chun Sung Bong (Caltech). Search for MFV RPV gluinos in dileptonic final states (\textit{2012}).
\item Max Horton (Caltech). Frequentist and Bayesian limit-setting for razor fit-based searches (\textit{2012}).
\end{itemize}

%----------------------------------------------------------------------------------------
%	COMPUTER SKILLS AND EXPERIENCE
%----------------------------------------------------------------------------------------

% \spacedlowsmallcaps{Computer Skills And Experience}\vspace{.5em}

% \Description{Programming Languages\ \ $\cdotp$\ \ \textsc{C++}, \textsc{python}, \LaTeX, \textsc{ROOT}, \textsc{RooFit}}

% \Description{Courses\ \ $\cdotp$\ \ Caltech CMS/CS/CNS/EE 155 Machine
%   Learning and Data Mining}

% %------------------------------------------------

% \vspace{.5em} % Extra space between major sections

%----------------------------------------------------------------------------------------
%	OTHER INFORMATION
%----------------------------------------------------------------------------------------

% \spacedlowsmallcaps{Other Information}\vspace{.5em}

% \newlength{\langbox} % Create a new length for the length of languages to keep them equally spaced
% \settowidth{\langbox}{English} % Length equals the length of "English" - if you have a longer language in your list put it here

% \Description{\MarginText{Languages}\textsc{English}
%   (Fluent) \ \ $\cdotp$\ \ \ \textsc{Spanish} (Fluent)}

% \vspace{.5em} % Negative vertical space to counteract the vertical space between every \Description command

% %------------------------------------------------

% \Description{\MarginText{Interests}Marathon Running (P.R. 3:15:18)\ \ $\cdotp$\ \ Soccer}

%----------------------------------------------------------------------------------------
% \cite{CMS-PAS-HIG-17-010}
% \cite{CMS-PAS-EXO-17-001}
% \cite{CMS-PAS-EXO-16-056}
% \cite{Duarte:2017bbq}
% \cite{CMS-PAS-EXO-16-032}
% \cite{CMS-PAS-SUS-16-045}
% \cite{CMS-PAS-SUS-15-004}
% \cite{razor8TeV}
% \cite{NIMA}


\clearpage
\nocite{*}

\MarginText{All Other Publications}

All other publications are listed here.

\vspace{10pt}
\printbibliography[heading=none,sorting=ynt,notkeyword={career}]

\end{resume}

\end{document}
